\documentclass{article}

\usepackage[style=alphabetic]{biblatex}
\addbibresource{type-theory.bib}

\usepackage{amsmath}
\usepackage{amssymb}
\usepackage{amsthm}

\newtheorem{theorem}{Theorem}[section]
\newtheorem{lemma}[theorem]{Lemma}
\newtheorem{example}[theorem]{Example}

\newcommand*{\Set}{\mathbf{Set}}
\newcommand*{\Ring}{\mathbf{Ring}}
\newcommand*{\CRing}{\mathbf{CRing}}
\newcommand*{\Graph}{\mathbf{Graph}}
\newcommand*{\Sh}{\mathrm{Sh}}
\newcommand*{\Psh}{\mathrm{Psh}}
\newcommand*{\T}{\mathbb{T}}
\newcommand*{\B}{\mathbb{B}}
\newcommand*{\Z}{\mathbb{Z}}
\newcommand*{\C}{\mathcal{C}}

% https://tex.stackexchange.com/a/223246
\newcommand*{\wand}{\mathbin{-\mkern-6mu*}}

\title{Instantiating Bunched Type Theory for Monoidal Classifying Toposes}
\author{finegeometer}

\begin{document}
\maketitle

\section{Setup}

Let \(\T\) be a geometric theory.

Suppose that for each pair \(A,B\) of sorts of \(\T\),
we have a relation \(\sim : A \times B \to \mathbf{Prop}\), geometrically defined in terms of \(\T\)'s signature.
For instance, if \(\T\) is the theory of rings, then \(\sim : R \times R \to \mathbf{Prop}\)
might be defined as \(a \sim b \iff ab = ba\).

For each sort \(A\), define the shorthand \(A_{\sim x} = \{a : A \mid a \sim x\}\).

We want to require these subsets to themselves form a \(\T\)-model.
If \(\T\) is the theory of rings, for instance, we want \(R_{\sim x}\) to be a ring, for any \(x : R\).

To state this precisely, say that \(\sim\) \emph{cuts out a sub-\(\T\)-model} if the following two conditions hold.
\begin{itemize}
    \item For any function symbol \(f : A_1 \times \dots \times A_n \to B\) and any \(x : X\),
    the restriction \(f_{\sim x} : (A_1)_{\sim x} \times \dots \times (A_n)_{\sim x} \to B_{\sim x}\) is well-defined for all \(x\).
    \item For any \(x : X\), every axiom of \(\T\) continues to hold when each sort \(A\) is replaced by \(A_{\sim x}\), each function symbol \(f\) is replaced by \(f_{\sim x}\), and each relation symbol is replaced by its restriction.
\end{itemize}

Let's look at the example where \(\T\) is the theory of \emph{local} rings,
where we define \(a \sim b \iff ab = ba\).

The first condition expands to the following five sequents.
\begin{align*}
    x : R, a : R, ax = xa, b : R, bx = xb &\vdash (a+b)x = x(a+b)
    \\ x : R, a : R, ax = xa, b : R, bx = xb &\vdash (ab)x = x(ab)
    \\ x : R, a : R, ax = xa &\vdash (-a)x = x(-a)
    \\ x : R &\vdash 0x = x0
    \\ x : R &\vdash 1x = x1
\end{align*}
In other words, it says that \(R_{\sim x} = \{a : R \mid a \sim x\}\) is a subring of \(R\), for all \(x : R\).

The second condition generates one requirement per axiom in the theory, but most of these are trivial.
The interesting one comes from the following axiom of local rings.
\begin{align*}
    a : R, b : R, (\exists c : R, (a+b)c = 1) &\vdash (\exists c : R, ac = 1) \lor (\exists c : R, bc = 1)
\end{align*}

The requirement generated is as follows.
\begin{align*}
    &x : R, a : R, ax = xa, b : R, bx = xb, (\exists c : R, cx = xc \land (a+b)c = 1)
    \\ &\vdash (\exists c : R, cx = xc \land ac = 1) \lor (\exists c : R, cx = xc \land bc = 1)
\end{align*}
This is exactly the condition that the subring \(R_{\sim x} \subseteq R\) is local.

\section{Theorem}

\begin{theorem} \label{main}
    Let \(\T\) be a geometric theory.
    For each pair of sorts \(A,B\) of \(\T\), let \(\sim : A \times B \to \mathbf{Prop}\) be a geometrically-defined relation which cuts out a sub-\(\T\)-model.
    Further suppose \(\sim\) is symmetric; we have \(a : A, b : B, a \sim b \vdash b \sim a\) for any sorts \(A\) and \(B\).
    Then we can interpret the type theory \(\mathbf{BT}(*, 1, \Sigma, \Pi, \Pi^*)\), as defined in \cite[Section 5.1]{schopp}, in \(\Set[\T]\).
\end{theorem}

\section{Proof}

We begin by bringing in some results from \cite{schopp}.

\begin{lemma}
    The type theory \(\mathbf{BT}(*, 1, \Sigma, \Pi, \Pi^*)\) can be modeled in any \\
    \((*, 1, \Sigma, \Pi, \Pi^*)\)-type-category.
\end{lemma}

\begin{proof}
    \cite{schopp}, section 6.
\end{proof}

\begin{lemma}
    Let \(\B\) be a topos equipped with a strict affine symmetric monoidal closed structure \((*,\wand)\),
    such that \(- * A\) preserves pullbacks for all \(A\).
    Further suppose we have a canonical choice of pullback for every span in \(\B\).
    Then the family fibration over \(\B\) is a \((*, 1, \Sigma, \Pi, \Pi^*)\)-type-category.
\end{lemma}

\begin{proof}
    \cite{schopp}, section 3.5. Since we are in a topos, the condition that the monomorphism \(A * B \hookrightarrow A \times B\) is strong is trivial.
\end{proof}

Next, we specialize to sheaf toposes.

\begin{lemma}
    Let \((C,J)\) be a (small) site, such that \(C\) is finitely complete.
    In fact, suppose we have a canonical choice for pullbacks in \(C\).
    Let \(*\) be an affine symmetric monoidal structure on \(C\) that preserves pullbacks and covers.
    Then \(\Sh(C,J)\) satisfies the conditions of the above lemma.
\end{lemma}

\begin{proof}
    \(\Sh(C,J)\) is a Grothendieck topos, hence a topos.

    \vspace*{1em}\hrule\vspace*{1em}

    Theorem 4.3.2 from \cite{biering} says we get a monoidal structure \(\otimes^\Sh\) on \(\Sh(C,J)\)
    by transporting Day convolution across the adjunction \((\mathbf{a} \dashv i) : \Psh(C,J) \rightleftarrows \Sh(C,J)\).
    Its monoidal unit is the sheafification of the Yoneda embedding of \(*\)'s unit, which simplifies to the terminal object.
    Chasing definitions quickly shows \(\otimes^\Sh\) is symmetric.
    And corollary 4.3.10 from \cite{biering} says this monoidal structure is closed.

    In other words, \(\otimes^\Sh\) is an affine symmetric monoidal closed structure.
    We must show this is strict --- that the canonical map \(A \otimes^\Sh B \to A \times B\) is always a monomorphism.

    But, letting \(\otimes^\Psh\) represent Day convolution on \(\Psh(C)\),
    this map is just \(iA \otimes^\Psh iB \to iA \times iB\), restricted to sheaves.
    Expanding both \(\otimes^\Psh\) and \(\times\) as Day convolutions,
    this is a map, natural in \(c \in C^\mathrm{op}\), of the following type.
    \begin{scriptsize}\[\left(\int^{c_1,c_2 \in C} A(c_1) \times B(c_2) \times (c \xrightarrow{C} c_1 * c_2)\right) \to \left(\int^{c_1,c_2 \in C} A(c_1) \times B(c_2) \times (c \xrightarrow{C} c_1 \times c_2)\right)\]\end{scriptsize}

    This map is exactly what you'd expect it to be; it composes the \(c \xrightarrow{C} c_1 * c_2\) component
    with the canonical map \(c_1 * c_2 \xrightarrow{C} c_1 \times c_2\), and leaves everything else alone.
    So since that canonical map is mono, this is too.

    \vspace*{1em}\hrule\vspace*{1em}

    Finally, we have a canonical construction for pullbacks in \(\Sh(C,J)\), since we have one for \(C\).
\end{proof}

And finally, we set the site to a syntactic site, to understand classifying toposes.

\begin{lemma}
    Let \(\T\) be a geometric theory.
    For each pair of sorts \(A,B\) of \(\T\), let \(\sim : A \times B \to \mathbf{Prop}\) be a symmetric, geometrically-defined relation, which cuts out a sub-\(\T\)-model.
    Then the syntactic site for \(\T\) satisfies the conditions of the above lemma.
\end{lemma}

\begin{proof}
    Let us begin with a few shorthands.

    We'll allow ourselves to write sequents involving compound types, such as \(p : A \times B \vdash \pi_1 p \sim \pi_2 p\).
    These can be straightforwardly ``compiled out'' to ordinary sequents, such as \(a : A, b : B \vdash a \sim b\).

    Next, if we have a context \(\Gamma = (a_1 : A_1, \dots, a_n : A_n, \phi_1 \dots \phi_p)\),
    we'll reuse the name \(\Gamma\) for the type \(\{(a_1, \dots, a_n) : A_1 \times \dots \times A_n \mid \phi_1 \land \dots \land \phi_p\}\).
    If we have a substitution \(f : \Gamma \to \Delta\), there is then a natural way to define the term \(\gamma : \Gamma \vdash f\gamma : \Delta\).

    Finally, if \(\gamma = (a_1, \dots, a_n) : \Gamma\) and \(\delta = (b_1, \dots, b_m) : \Delta\),
    we write \(\gamma \sim \delta\) as a shorthand for \(\bigwedge_{i=1}^m \bigwedge_{j=1}^n a_i \sim b_j\).

    With that out of the way, let's begin the proof.

    \vspace*{1em}\hrule\vspace*{1em}

    First of all, the syntactic category is guaranteed to be finitely complete, with a canonical construction for pullbacks.
    Specifically, the empty context is terminal,
    and the pullback of the span \(\Gamma \xrightarrow{f} \Xi \xleftarrow{g} \Delta\) is the context \((\gamma : \Gamma, \delta : \Delta, f\gamma = g\delta)\).

    Second, we describe the monoidal structure. Given contexts \(\Gamma\) and \(\Delta\),
    define \(\Gamma * \Delta = (\gamma : \Gamma, \delta : \Delta, \gamma \sim \delta)\).

    Functoriality of \(*\) reduces to the fact that \(\sim\) respects substitutions, which further reduces to the fact that \(\sim\) respects function symbols.
    Symmetry reduces to the symmetry of \(\sim\). Choosing the empty context as monoidal unit, the unit laws hold on the nose, and associativity up to a rearrangement of contexts.
    All the coherences hold.
    And since \(\Gamma * \Delta\) is just \(\Gamma \times \Delta\) with extra propositional hypotheses,
    the map \(\Gamma * \Delta \hookrightarrow \Gamma \times \Delta\) is a monomorphism.
    Putting that all together, \(*\) is a strict affine symmetric monoidal product on the syntactic category.

    Third, we show that starring preserves pullbacks.
    
    Say we have a span \(\Gamma \xrightarrow{f} \Xi \xleftarrow{g} \Delta\), and another context \(\Theta\).
    The pullback of the span is the context \((\gamma : \Gamma, \delta : \Delta, f\gamma = g\delta)\).
    Starring with \(\Theta\) yields \((\gamma : \Gamma, \delta : \Delta, f\gamma = g\delta, \theta : \Theta, \gamma \sim \theta, \delta \sim \theta)\).
    
    On the other hand, if we star with \(\Theta\) first, we get the span
    \((\gamma : \Gamma, \theta : \Theta, \gamma \sim \theta) \xrightarrow{f * \mathrm{id}_\Theta} (\xi : \Xi, \theta : \Theta, \xi \sim \theta) \xleftarrow{g * \mathrm{id}_\Theta} (\delta : \Delta, \theta : \Theta, \delta \sim \theta)\).
    Taking the pullback yields \((\gamma : \Gamma, \theta_1 : \Theta, \gamma \sim \theta_1, \delta : \Delta, \theta_2 : \Theta, \delta \sim \theta_2, f\gamma = g\delta, \theta_1 = \theta_2)\),
    which is equivalent to what we got before.

    And fourth, we show that starring preserves covers.
    
    Suppose \(\{f_i : \Gamma_i \to \Delta \mid i \in I\} \in J\).
    This means \(\delta : \Delta \vdash \bigvee_{i \in I} \exists \gamma : \Gamma_i, f_i\gamma = \delta\) is provable.

    By the final hypothesis, the proof is still valid if we, throughout the proof, replace each sort \(X\) with \(X_{\sim \theta} = \{x : X \mid x \sim \theta\}\), and each function \(f\) with its restriction \(f_{\sim \theta}\).
    This is only stated when \(\theta : \Theta\) for some \emph{sort} \(\Theta\),
    but the same is true for any \emph{context} \(\Theta\),
    simply by iterating once for each variable in \(\Theta\).

    So \(\theta : \Theta, \delta : \Delta_{\sim \theta} \vdash \bigvee_{i \in I} \exists \gamma : (\Gamma_i)_{\sim \theta}, f_i\gamma = \delta\).
    But this is equivalent to the statement \(\delta : (\Delta * \Theta) \vdash \bigvee_{i \in I} \exists \gamma : (\Gamma_i * \Theta), (f_i * \mathrm{id}_\Theta)\gamma = \delta\),
    so \(\{f_i * \mathrm{id}_\Theta : \Gamma_i * \Theta \to \Delta * \Theta \mid i \in I\}\) is a cover, as required.
\end{proof}

Putting this all together:

\begin{proof}[Proof of Theorem \ref{main}]
    Combine the above four lemmas.
\end{proof}

\printbibliography

\end{document}